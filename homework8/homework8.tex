\documentclass{article}
\usepackage{enumerate}
\usepackage{amsmath}
\usepackage{amssymb}
\usepackage{graphicx}
\usepackage{subfigure}
\usepackage{geometry}
\geometry{left=3.0cm,right=3.0cm,top=4.0cm,bottom=4.0cm}
\renewcommand{\thesection}{Problem \arabic{section}.}
\begin{document}

\section{}
\begin{enumerate}[(a)]
	\item
	$$f=3N-m=5$$
	There is one constraint: the fixed distance between two objects.
	\item
	$$f=3N-m=6$$
	There are no constraints.
	\item
	$$f=3N-m=6$$
	There are three constraints: the fixed distance between every two objects.
	\item
	$$f=3N-m=9$$
	There are no constraints.
\end{enumerate}

\section{}
\begin{enumerate}[(a)]
	\item
	Suppose a point on the incident ray $(x_1,y_1)$, a point on the reflection ray $(x_2,y_2)$ and the speed of rays $v$.
	$$t=\frac{\sqrt{(x_1-x)^2+y_1^2}}{v}+\frac{\sqrt{(x_2-x)^2+y_2^2}}{v}$$
	where $(x,0)$ is the intersection point of two rays.\\
	When t get the minimum value,	
	$$\frac{dt}{dx}=\frac{x-x_1}{\sqrt{(x_1-x)^2+y_1^2}v}+\frac{x-x_2}{\sqrt{(x_2-x)^2+y_2^2}v}=0$$
	$$\frac{\sin\alpha}{v}=\frac{\sin\beta}{v}$$
	$$\alpha=\beta$$
	where $\alpha,\ \beta$ is the angle between the incident ray, reflection ray and the y-axis.
	\item
	Suppose a point on the incident ray $(x_1,y_1)$, a point on the refraction ray $(x_2,y_2)$ and the speed of each ray $v_1,\ v_2$.
	$$t=\frac{\sqrt{(x_1-x)^2+y_1^2}}{v_1}+\frac{\sqrt{(x_2-x)^2+y_2^2}}{v_2}$$
	where $(x,0)$ is the intersection point of two rays.\\
	When t get the minimum value,	
	$$\frac{dt}{dx}=\frac{x-x_1}{\sqrt{(x_1-x)^2+y_1^2}v_1}+\frac{x-x_2}{\sqrt{(x_2-x)^2+y_2^2}v_2}=0$$
	$$\frac{\sin\alpha}{v_1}=\frac{\sin\beta}{v_2}$$
	where $\alpha,\ \beta$ is the angle between the incident ray, refraction ray and the y-axis.
\end{enumerate}

\section{}
\begin{enumerate}[(a)]
	\item
	$$f=3N-m=2$$
	The generalized coordinates are the angle $\theta$ between the pendulum and z-axis and the rotating angle $\varphi$ on the xy-plane.
	\item
	Suppose the center of the sphere to be the zero potential point.
	$$L=K-U=\frac{1}{2}mv^2-(-mgR\cos\theta)=\frac{1}{2}mR^2(\dot{\theta}^2+\sin^2\theta\dot{\varphi}^2)+mgR\cos\theta$$
	\item
	\begin{eqnarray*}
	\left\{
		\begin{array}{ll}
		\frac{d}{dt}\left(\frac{\partial L}{\partial\dot{\theta}}\right)-\frac{\partial L}{\partial\theta}&=\ 0\\
		\frac{d}{dt}\left(\frac{\partial L}{\partial\dot{\varphi}}\right)-\frac{\partial L}{\partial\varphi}&=\ 0\\
		\end{array}
	\right.\Longrightarrow\left\{
		\begin{array}{ll}
		mR^2\ddot{\theta}-mR^2\sin\theta\cos\theta\dot{\varphi}^2+mgR\sin\theta&=\ 0\\
		mR^2\sin^2\theta\ddot{\varphi}+2mR^2\sin\theta\cos\theta\dot{\theta}\dot{\varphi}&=\ 0\\
		\end{array}
	\right.
	\end{eqnarray*}
	$$\ddot{\theta}=\sin\theta\cos\theta\dot{\varphi}^2-\frac{g\sin\theta}{R}$$
	$$\ddot{\varphi}=-2\cot\theta\dot{\theta}\dot{\varphi}$$
\end{enumerate}

\section{}
	Suppose M$(x_1,0)$, m$(x2,H-(x1-x2)\tan\theta)$, the ground to be the zero potential plane and left to be the positive direction.
	\begin{align*}
	L&=K-U=\frac{1}{2}M\dot{x_1}^2+\frac{1}{2}m[\dot{x_2}^2+(\dot{x_1}-\dot{x_2})^2\tan\theta^2]-H+mg((x1-x2)\tan\theta)\\
	&=\frac{1}{2}(M+m\tan^2\theta)\dot{x_1}^2+\frac{1}{2}(m+m\tan^2\theta)\dot{x_2}^2-m\tan^2\theta\dot{x_1}\dot{x_2}-H+mg(x_1-x_2)\tan\theta
	\end{align*}
	\begin{eqnarray*}
	\left\{
		\begin{array}{ll}
		\frac{d}{dt}\left(\frac{\partial L}{\partial\dot{x_1}}\right)-\frac{\partial L}{\partial x_1}&=\ 0\\
		\frac{d}{dt}\left(\frac{\partial L}{\partial\dot{x_2}}\right)-\frac{\partial L}{\partial x_2}&=\ 0\\
		\end{array}
	\right.\Longrightarrow\left\{
		\begin{array}{ll}
		(M+m\tan^2\theta)\ddot{x_1}-m\tan^2\theta\ddot{x_2}-mg\tan\theta&=\ 0\\
		(m+m\tan^2\theta)\ddot{x_2}-m\tan^2\theta\ddot{x_1}+mg\tan\theta&=\ 0\\
		\end{array}
	\right.
	\end{eqnarray*}
	$$\ddot{x_2}=\frac{m\tan^2\theta\ddot{x_1}-mg\tan\theta}{m+m\tan^2\theta}=\frac{\tan^2\theta\ddot{x_1}-g\tan\theta}{1+\tan^2\theta}$$
	$$(M+m\tan^2\theta)\ddot{x_1}-m\tan^2\theta\frac{\tan^2\theta\ddot{x_1}-g\tan\theta}{1+\tan^2\theta}-mg\tan\theta=0$$
	$$\ddot{x_1}=\frac{-mg\tan^3\theta+(1+\tan^2\theta)mg\tan\theta}{(1+\tan^2\theta)(M+m\tan^2\theta)-m\tan^4\theta}=\frac{mg\tan\theta}{M+M\tan^2\theta+m\tan^2\theta}$$

\section{}
	\begin{align*}
	\bar{x}&=\frac{\sum_{i=1}^3m_ix_i}{\sum_{i=1}^3m_i}=\frac{7}{6}cm\\
	\bar{y}&=\frac{\sum_{i=1}^3m_iy_i}{\sum_{i=1}^3m_i}=2cm\\
	\bar{z}&=\frac{\sum_{i=1}^3m_iz_i}{\sum_{i=1}^3m_i}=\frac{17}{6}cm
	\end{align*}
	$$a=\frac{F}{m}=\frac{0.05}{0.03}=\frac{5}{3}m/s^2$$
	$$\Delta x=\frac{1}{2}at^2=\frac{10}{3}m$$
	$$x+\Delta x=\frac{669}{2}cm$$
	So the position of the mass center is $(\frac{669}{2},2,\frac{17}{6})$ cm
	
\section{}
\begin{enumerate}[(a)]
	\item
	$$m(x)=\rho A$$
	$$\bar{x}=\frac{\int_0^lxm(x)dx}{\int_0^lm(x)dx}=\frac{\frac{1}{2}\rho Al^2}{\rho Al}=\frac{1}{2}l$$	
	\item
	$$m(x)=\rho A=\alpha Ax$$
	$$\bar{x}=\frac{\int_0^lxm(x)dx}{\int_0^lm(x)dx}=\frac{\frac{1}{3}\alpha Al^3}{\frac{1}{2}\alpha Al^2}=\frac{2}{3}l$$
\end{enumerate}
	
\section{}
\begin{enumerate}[(a)]
	\item
	Since the half-circle is symmetrical about the y-axis,
	$$\bar{x}=0$$
	According to Pappus’ centroid theorem, $S=Lr$
	$$4\pi R^2=\pi R\cdot2\pi\bar{y}$$
	$$\bar{y}=\frac{2R}{\pi}$$
	\item
	Since the half-cylinder is symmetrical about the y-axis,
	$$\bar{x}=0$$
	Since it is a half-cylinder,
	$$\bar{z}=\frac{1}{2}a$$
	According to Pappus’ centroid theorem, $V=Sr$
	$$\frac{4}{3}\pi R^3=\frac{1}{2}\pi R^2\cdot2\pi\bar{y}$$
	$$\bar{y}=\frac{4R}{3\pi}$$
\end{enumerate}

\section{}
	The mass center of the system won't move.\\
	Suppose the direction he moved to be the positive direction, the displacement of the fisherman and the boat to be $x_1,x_2$
	\begin{eqnarray*}
	\left\{
		\begin{array}{ll}
		x_1-x_2&=l\\
		mx_1+Mx_2&=0\\
		\end{array}
	\right.\Longrightarrow
		x_2=-\frac{ml}{M+m}
	\end{eqnarray*}
	So the distance the boat has moved with respect to the bank is $\frac{ml}{M+m}$

\section{}
	Suppose the speed direction to be positive.\\
	the change momentum of the middle boat is
	$$\Delta P=mu-mu=0$$
	So the speed of the middle boat won't change.\\
	In the system of the first boat and the thrown object,
	$$Mv+m(v+u)=(M+m)v_1$$
	$$v_1=\frac{Mv+m(v+u)}{M+m}=v+\frac{mu}{M+m}$$
	In the system of the last boat and the thrown object,
	$$Mv+m(v-u)=(M+m)v_2$$
	$$v_2=\frac{Mv+m(v-u)}{M+m}=v-\frac{mu}{M+m}$$

\section{}
	After the collision, the equilibrium point will change by $\Delta l$
	$$Mg=kl$$
	$$(M+m)g=k(l+\Delta l)$$
	$$\Delta l=\frac{mg}{k}$$
	The speed of the system will be $v'$
	$$v=\sqrt{2gh}$$
	$$mv=(M+m)v'$$
	$$v'=\frac{m\sqrt{2gh}}{M+m}$$
	According to the law of conservation of energy,
	$$\frac{1}{2}kA^2=\frac{1}{2}(M+m)v'^2+\frac{1}{2}k\Delta l^2$$
	$$A=\sqrt{\frac{2m^gh}{k(M+m)}+\frac{m^2g^2}{k}}=\sqrt{\frac{m^2l}{M}\left(\frac{2h}{M+m}+\frac{l}{M}\right)}$$
	
\section{}
	At first, two blocks move left together until the left block collides with the wall.
	$$t_1=\frac{L}{v_0}$$
	Then the left block collides with the wall so that it turns right at the same speed $v_0$. There happens a harmonic oscillation between two blacks until the left block collides with the wall again.
	$$t_2=\frac{1}{2}T=\pi\sqrt{\frac{m}{2k}}$$
	After that the left block turns right again and the right block also moves right until they return to the origin position.
	$$t_3=\frac{L}{v_0}$$
	$$t=t_1+t_2+t_3=\frac{2L}{v_0}+\pi\sqrt{\frac{m}{2k}}$$
	
\section{}
	Suppose left to be the positive direction
	$$mv=(m-dm)(v+dv)+(v+u)dm$$
	$$mdv+udm=0$$
	$$dv=-\frac{u}{m}dm$$
	Do integral on both sides,
	$$\int_0^vdv=-u\int_m^{m-\alpha T}\frac{1}{m}dm$$
	$$v=uln\frac{m}{m-\alpha T}$$
	After the collision, we can also find
	$$mv=(m-dm)(v+dv)+(v+u)dm$$
	$$v=uln\frac{m-\alpha T}{m-\alpha (T+t)}$$
	$$\frac{m}{m-\alpha T}=\frac{m-\alpha T}{m-\alpha (T+t)}$$
	$$m-\alpha (T+t)=\frac{(m-\alpha T)^2}{m}$$
	$$t=\frac{1}{\alpha}\left[m-\frac{(m-\alpha T)^2}{m}\right]-T$$
	When $m=1kg$, $\alpha=0.01kg/s$, $T=10s$,
	$$t=9s$$
\end{document}

