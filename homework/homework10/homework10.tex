\documentclass{article}
\usepackage{enumerate}
\usepackage{amsmath}
\usepackage{amssymb}
\usepackage{graphicx}
\usepackage{subfigure}
\usepackage{geometry}
\geometry{left=3.0cm,right=3.0cm,top=3.0cm,bottom=4.0cm}
\renewcommand{\thesection}{Problem \arabic{section}.}
\begin{document}

\section{}
\begin{eqnarray*}
\left\{
\begin{array}{l}
I=m_1r_1^2+m_2r_2^2\\
m_1r_1=m_2r_2\\
\end{array}
\right.\quad\Longrightarrow\quad
\left\{
\begin{array}{ll}
r_1&=\sqrt{\frac{m_2I}{m_1^2+m_1m_2}}\\
r_2&=\sqrt{\frac{m_1I}{m_2^2+m_1m_2}}\\
\end{array}
\right.
\end{eqnarray*} 

$$H=r_1+r_2=\sqrt{\frac{I}{m_1+m_2}}\left(\sqrt{\frac{m_2}{m_1}}+\sqrt{\frac{m_1}{m_2}}\right)=9.29\times10^{-11}m$$

\section{}
\begin{enumerate}[(a)]
\item
	$$I=\int_M r^2dm=\int_{R_1}^{R_2}2\alpha\pi Hr^4dr=\frac{2}{5}\alpha\pi H(R_2^2-R_1^2)$$
	
\item
	$$I=\int_M r^2dm=\int_0^h\frac{\sigma ar^2(h-r)}{h}dr=\frac{1}{12}\sigma ah^3$$

\item	
	$$I=\int_M r^2dm=\int_0^h\frac{\sigma ar^3}{h}dr=\frac{1}{4}\sigma ah^3$$

\end{enumerate}

\section{}

\begin{align*}
I_x&=\int_M(x^2+z^2)dm\approx\int_Mx^2dm\\
I_y&=\int_M(y^2+z^2)dm\approx\int_My^2dm\\
I_z&=\int_M(x^2+y^2)dm=I_x+I_y
\end{align*}

\section{}
Suppose the equation of the parabola is
$$z=ar^2+C$$
$$V_0=\int_VdV=\int_0^R2\pi r(ar^2+C)dr=\frac{1}{2}a\pi R^4+C\pi R^2$$
Considering a point on the parabola,
$$mg\tan\theta=m\omega^2$$
$$\tan\theta=\frac{\omega^2}{g}=z'=2a$$
$$a=\frac{\omega^2}{2g}$$
$$C=\frac{V_0-\frac{1}{2}a\pi R^4}{\pi R^2}=\frac{V_0}{\pi R^2}-\frac{\omega^2R^2}{4g}$$
$$z=\frac{\omega^2}{2g}r^2+\frac{V_0}{\pi R^2}-\frac{\omega^2R^2}{4g}$$

\begin{align*}
	I_w&=\int_M r^2dm=\int_0^R2\rho\pi r^3(ar^2+C)dr=2\rho\pi\left(\frac{1}{6}aR^6+\frac{1}{4}CR^4\right)\\
	&=2\rho\pi\left[\frac{1}{6}\frac{\omega^2}{2g}R^6+\frac{1}{4}\left(\frac{V_0}{\pi R^2}-\frac{\omega^2R^2}{4g}\right)R^4\right]\\
	&=\frac{1}{24}\rho\pi\omega^2R^6+\frac{1}{2}\rho V_0R^2
\end{align*}

$$E_k=\frac{1}{2}(I_0+I_w)\omega^2=\frac{1}{2}I_0\omega^2+\frac{1}{48}\rho\pi\omega^4R^6+\frac{1}{4}\rho V_0\omega^2R^2$$

\section{}
$$\frac{1}{2}mv^2+\frac{1}{2}I\omega^2+\frac{1}{2}kx^2=C$$
$$\omega=\frac{v}{R}$$
$$\frac{1}{2}mv^2+\frac{1}{2}I\frac{v^2}{R^2}+\frac{1}{2}kx^2=C$$
$$\frac{1}{2}\left(m+\frac{I}{R^2}\right)v^2+\frac{1}{2}kx^2=C$$

According to the equation of harmonic oscillation,
$$T=2\pi\sqrt{\frac{m+\frac{I}{R^2}}{k}}=2\pi\sqrt{\frac{mR^2+I}{kR^2}}$$

\section{}
\begin{enumerate}[(a)]
\item the left one:\\
The moment of inertia about the axis perpendicular to the square and passing through the center of the square is
$$I_c=\frac{1}{6}mL^2$$
The moment of inertia about the axis perpendicular to the hanging point is
$$I=I_c+md^2=\frac{1}{6}mL^2+\frac{9}{4}mL^2=\frac{29}{12}mL^2$$
$$T=2\pi\sqrt{\frac{I}{mgd}}=2\pi\sqrt{\frac{29L}{18g}}$$

\item the right one:\\
The edge of the square won't rotate, so
$$F=-mg\tan\theta=-\frac{mg}{L}x$$
$$T=2\pi\sqrt{\frac{k}{m}}=2\pi\sqrt{\frac{L}{g}}$$

\end{enumerate}

\section{}
Suppose the moving direction to be the positive direction and the rolling direction to be the positive angular direction.
\begin{enumerate}[(a)]
\item
\vspace{7cm}
$$a_x=\frac{F}{M}=\frac{\mu_kMg}{M}=\mu_kg$$
$$\varepsilon_z=-\frac{RF}{I}=-\frac{\mu_kMgR}{\frac{1}{2}MR^2}=-\frac{2\mu_kg}{R}$$
\item
$$\mu_kgt=\left(\omega_0-\frac{2\mu_kg}{R}t\right)R$$
$$t=\frac{\omega_0R}{3\mu_kg}$$
$$x=\frac{1}{2}a_xt^2=\frac{\omega_0^2R^2}{18\mu_kg}$$
\item
$$x'=v_0t-\frac{1}{2}at^2=\omega_0Rt-\frac{1}{2}\varepsilon_zRt^2=\frac{2\omega_0^2R^2}{9\mu_kg}$$
$$W=-fx'=-\frac{2}{9}M\omega_0^2R^2$$
\end{enumerate}

\section{}
$$F=ma=f-mg\sin\theta$$
$$fR=I\varepsilon$$
$$a=R\varepsilon=\frac{fR^2}{I}=\frac{m(g\sin\theta-a)R^2}{I}$$
$$a=\frac{mgR^2\sin\theta}{mR^2+I}$$
$$I_b=\frac{2}{5}mR^2\quad a_b=\frac{5}{7}g\sin\theta$$
$$I_r=mR^2\quad a_r=\frac{1}{2}g\sin\theta$$
$$\frac{1}{2}a_bt^2=v_0t+\frac{1}{2}a_lt^2$$
$$v_0=\frac{1}{2}t(a_b-a_l)=\frac{3}{28}gt\sin\theta$$

\section{}
$$m_1=\rho a\pi r^2$$
$$m_2=\rho a\pi R^2$$
$$I=\frac{1}{2}m_1r^2+m_2R^2=\frac{1}{2}\rho a\pi r^4+\rho a\pi R^4=\rho a\pi\left(\frac{1}{2}r^4+R^4\right)$$
$$m=m_1+2m_2=\rho a\pi(r^2+2R^2)$$
$$F=ma=mg-f$$
$$a=R\varepsilon=\frac{fr^2}{I}=\frac{m(a-g)r^2}{I}$$
$$a=\frac{mgr^2}{mr^2+I}=\frac{\rho a\pi(r^2+2R^2)gr^2}{\rho a\pi(r^2+2R^2)r^2+\rho a\pi\left(\frac{1}{2}r^4+R^4\right)}=\frac{3r^2(r^2+2R^2)}{3r^4+4R^2r^2+2R^4}$$

\section{}
$$F=ma=mg\sin\alpha-f-\mu mg\cos\alpha$$
$$fr-\mu mgR\cos\alpha=I_0\varepsilon$$
$$a=r\varepsilon=\frac{r(fr-\mu mgR\cos\alpha)}{I_0}
=\frac{(mg\sin\alpha-ma-\mu mg\cos\alpha)r^2-\mu mgrR\cos\alpha}{I_0}$$
$$a=\frac{mg[r^2\sin\alpha-\mu(r^2+R^2)\cos\alpha]}{mr^2+I_0}$$

\end{document}
