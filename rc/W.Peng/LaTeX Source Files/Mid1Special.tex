\documentclass{beamer}
\usepackage{bbm,color,xcolor,times,graphicx,graphics,fancyhdr,listings,subfigure,lastpage,setspace,algorithm2e,mathtools,amsmath,amsfonts,float,animate,multicol}
\title{VP160 Honors Physics I Midterm 1 Review\\Simple Harmonic Oscillation}
\setbeamertemplate{navigation symbols}{}%remove navigation symbols
\author{W.Peng}
\institute{UM-SJTU JI}
\date{Summer 2016}
\subject{Physics}
\begin{document}
\newcommand{\unit}[1]{\text{ }\mathrm{#1}}
\newcommand{\derivative}{\mathrm{d}}
\begin{frame}
\titlepage
\end{frame}
\begin{frame}
\frametitle{Solutions to the Mathematical Problem}
To identify a motion as a \alert{simple harmonic oscillation}, the goal is to conclude that the equation of motion is
\[\frac{\derivative^2 x}{\derivative t^2}+\frac{k}{m}x=0\]
The general solution is \[x(t)=A\cos(\omega_0 t+\varphi)\]
where $A$ is the amplitude, $\omega_0$ is the natural angular frequency, and $\varphi$ is the initial phase. $\omega_0=\sqrt{\frac{k}{m}}$. Period $T=\frac{2\pi}{\omega_0}$
\[v(t)=-A\omega_0\sin(\omega_0 t+\varphi)\quad a(t)=-A\omega_0^2\cos(\omega_0 t+\varphi)\]
It is useful to notice that \[v^2+(\omega_0 x)^2=(\omega_0 A)^2\]
\end{frame}
\begin{frame}
\frametitle{Analogy to a Uniform Circular Motion}
Now consider a tray with mass $M$ and a ball with mass $m$. A spring with spring constant $k$ is attached to the ground on one end. Now the ball is placed in the tray, and the tray is attached to the spring. Suppose the tray is pushed downwards by $A$ from equilibrium ($A>\frac{(M+m)g}{k}$) and is released from rest. Find the \alert{position} at which the ball is about to leave the tray, the \alert{time} it takes for the tray and the ball to reach this position, and the \alert{velocity} of the ball at that position.\\
\begin{block}{Solution}
The ball will leave the tray when the spring reaches its \alert{original length} because the tray can only exert a push on the ball. When the spring is extended, the spring will be exerting pull on the tray, causing its acceleration to be greater than \alert{acceleration due to gravity}, greater than the maximal downward acceleration the ball can have. Therefore, the displacement from equilibrium is
\[s=\frac{(M+m)g}{k}\]
\end{block}
\end{frame}
\begin{frame}
The time it takes the tray to travel to that position can be found by finding the \alert{phase} the circular motion has traversed. Before the ball leaves the tray, $\omega_0=\sqrt{\frac{M+m}{k}}$
\[x(t)=A\cos(\omega_0 t-\pi)\]
Therefore, at that position, the phase is $\theta=-\arccos\frac{s}{A}$, so the time $t$ it takes satisfies \[\omega_0 t=\theta-\varphi=\pi-\arccos\frac{s}{A}=\pi-\arccos\frac{(M+m)g}{kA}\]
Finally, the velocity $v_t$ at that position satisfies
\[v_t^2+(\omega_0 s)^2=(\omega_0 A)^2\]
\[v_t=\sqrt{\frac{M+m}{k}\left[A^2-\left(\frac{(M+m)g}{k}\right)^2\right]}\]
\end{frame}
\begin{frame}
\frametitle{Series Expansion}
There are cases where the oscillation does not have a linear restoring force, but when the amplitude is small, the oscillation can be \alert{approximated} as a harmonic oscillation. To do so, it is sometimes useful to know the series expansion of $f$ around $x_0$: $f(x)=\sum_{k=0}^{\infty}a_{k}(x-x_0)^k$
\[f(x)=a_0+a_1(x-x_0)+a_2(x-x_0)^2+a_3(x-x_0)^3+\dots\]
\[f'(x)=a_1+2a_2(x-x_0)+3a_3(x-x_0)^2+4a_4(x-x_0)^3+\dots\]
\[f''(x)=2a_2+6a_3(x-x_0)+12a_4(x-x_0)^2+20a_5(x-x_0)^3+\dots\]
Our goal is to determine $a_n$, and in fact we can calculate $a_n$ by \alert{differentiating} both sides $n$ times and \alert{taking the value} at $x_0$.\\
$f(x_0)=a_0$; $f'(x_0)=a_1$; $f''(x_0)=2a_2$; $f'''(x_0)=6a_3$. In general, \[f^{(n)}(x_0)=\alert{n!}a_n\implies a_n=\frac{f^{(n)}(x_0)}{\alert{n!}}\]
We keep $F$ to the \alert{linear} term, or $U$ to the \alert{quadratic} term.
\end{frame}
\begin{frame}
\frametitle{Spring Mass System with a Nonlinear Spring}
Suppose we have a mass spring system with mass $m$ and a nonlinear spring \[F(x)=x^{\alpha}(1-\frac{x}{l_0})^{\beta}\]We would like to put $F$ into a series expansion around $l_0$, i.e., $F(x)=\sum_{k=0}^{\infty}a_k(x-l_0)^k$. Now $a_0=F(l_0)=0$, \[a_1=\left.\frac{\derivative F}{\derivative x}\right|_{x=l_0}=\left.\alpha x^{\alpha-1}(1-\frac{x}{l_0})^\beta+x^\alpha\beta(1-\frac{x}{l_0})^{\beta-1}(-\frac{1}{l_0})\right|_{x=l_0}\]
we only consider $\beta=1$, in which case $a_1=l_0^\alpha(-\frac{1}{l_0})$, \[F(x)\approx - l_0^{\alpha-1}(x-l_0)\]\[\omega=\sqrt{\frac{l_0^{\alpha-1}}{m}}\]
\end{frame}
\begin{frame}
\frametitle{Oscillation in a Potential Well}
A mass $M$ is in a potential \alert{well} $V(x)=-kxe^{-ax}$ ($k,a>0$,constants) along the $x$ axis. Find the equilibrium position and the period of oscillation with small amplitude around the equilibrium.\\
Now $F(x)=-\frac{\derivative V}{\derivative x}=ke^{-ax}+kx(-a)e^{-ax}$, and the equilibrium position is identified at $F(x)=0$, so $x_0=\frac{1}{a}$. Furthermore, $\left.\frac{\derivative F}{\derivative x}\right|_{x=\frac{1}{a}}=\left.k(-a)e^{-ax}-a(ke^{-ax}+kx(-a)e^{-ax})\right|_{x=\frac{1}{a}}=-kae^{-1}-a(ke^{-1}-ke^{-1})=-\frac{ka}{e}$, so
\[F\approx-\frac{ka}{e}(x-\frac{1}{a})\]
\[\omega=\sqrt{\frac{ka}{eM}}\]
\end{frame}
\begin{frame}
\frametitle{Oscillation in Center of Mass FoR}
Suppose mass $m_1$, $m_2$ are placed on a frictionless horizontal surface. They are connected by a spring with constant $k$. Find the angular frequency of oscillation.\\
Now the center of mass of these two masses is static, so the center-of-mass FoR is an inertial frame of reference. Furthermore, the spring can be seen as two segments, one on the left, connected to $m_1$, with spring constant $k_1=\frac{m_1+m_2}{m_2}k$, and the other segment on the right, connected to $m_2$, with spring constant $k_2=\frac{m_1+m_2}{m_1}k$, so the angular frequency $\omega=\sqrt{\frac{m_1+m_2}{m_1m_2}k}$
where $\frac{m_1m_2}{m_1+m_2}$ is the \alert{reduced mass} of two bodies.
\end{frame}
\begin{frame}
\frametitle{Reduced Mass of Two Bodies}
Suppose there are two bodies 1, 2 under mutual interaction $F$. Then the acceleration of 2 is $\overline a_2=\frac{F}{m_2}\hat n_{12}$, and $\overline a_1=\frac{F}{m_1}\hat n_{21}$, so the relative acceleration of 2 to 1 is
\[\overline a_2-\overline a_1=\hat n_{12}\left(\frac{F}{m_2}+\frac{F}{m_1}\right)=\hat n_{12}\frac{F}{\frac{m_1m_2}{m_1+m_2}}\]
so it is as if 1 is fixed and 2 has \alert{reduced mass} of $\frac{m_1m_2}{m_1+m_2}$.
\end{frame}
\begin{frame}
\frametitle{Maximum Angular Frequency of Vertical Oscillator}
Suppose a block is placed on a vertical plane which is attached to a spring in a simple harmonic oscillation with amplitude $A$. What is the maximum angular frequency so that the block is always in contact with the plane?\\
The maximum acceleration the block can have is $g$. The maximum acceleration of the oscillator is $\omega^2 A$. Therefore, the maximum allowed angular frequency $\omega_{max}=\sqrt{\frac{g}{A}}$.
\end{frame}
\begin{frame}
\frametitle{Oscillation of Liquid}
\alert{Caution} The normal forces provided by the wall of the tube is unknown. We study the forces \alert{along} the tube.\\
Find the natural angular frequency of small oscillations in a V-shaped tube of constant cross-sectional area and length of the liquid column $l$. The arms of the tube are inclined at angles $\alpha$ and $\beta$ to the horizontal. Neglect viscosity.\\
Assume from equilibrium, the liquid flows along the tube by $x$ to the right. This difference in height will result in a difference in pressure which will generate a force along the tube that serves as the restoring force. Assuming the buck density of the liquid is $\rho$, the constant area of a cross section of the tube is $S$, then the pressure difference is
\[P=-\rho Sxg(\sin\beta+\sin\alpha)\]
so the restoring force is $F=PS$, and the mass of the liquid is $\rho Sl$, so the angular frequency is $\omega=\sqrt{\frac{g(\sin\beta+\sin\alpha)}{l}}$
\end{frame}
\begin{frame}
\frametitle{Free Fall from a High Altitude}
\begin{block}{Question}
A particle falls on the Earth from a \alert{high altitude} $h$. Neglecting air drag, find the time $T$ when it hits the ground and the speed it has at this instant.
\end{block}
\begin{block}{Solution}
$a=\alert{-}GM/x^2$ where $x_{\text{initial}}=h+R$ and $x_{\text{final}}=R$. $(\derivative v/\derivative t)\derivative x=-GM/x^2 \derivative x$ Now using $\derivative x=v\derivative t$, $GM=gR^2$,
\[v\derivative v=-\frac{gR^2}{x^2}\derivative x\overset{\text{Integrating both sides}}{\implies} \frac{1}{2}v_{\text{final}}^2=\frac{gR^2}{R}-\frac{gR^2}{h+R}\implies \]$v_{\text{final}}=\sqrt{2gR^2\left(\frac{1}{R}-\frac{1}{R+h}\right)}$
\end{block}
\end{frame}
\begin{frame}
\frametitle{Free Fall from a High Altitude (Continued)}
Now at other altitudes, the speed can be determined in an identical fashion.
$v=\sqrt{2gR^2\left(\frac{1}{x}-\frac{1}{R+h}\right)}$ Therefore,
$\derivative t=\frac{-\derivative x}{v}$
\[
T=-\int_{R+h}^{R}\frac{\derivative x}{\sqrt{2gR^2\left(\frac{1}{x}-\frac{1}{R+h}\right)}}=\frac{1}{\sqrt{2gR^2}}\int_R^{R+h}\frac{\sqrt{x}\derivative x}{\sqrt{1-\frac{x}{R+h}}}
\]
Now setting $s=\sqrt{\frac{x}{R+h}}$, so $x=(s^2)(R+h)$, $\derivative x=2s(R+h)\derivative s$
\[
T=\frac{1}{\sqrt{2gR^2}}\int_{\sqrt{\frac{R}{R+h}}}^{1}\frac{s\sqrt{R+h}2s(R+h)\derivative s}{\sqrt{1-s^2}}\]\[=\frac{2(R+h)^{3/2}}{\sqrt{2gR^2}}\int_{\sqrt{\frac{R}{R+h}}}^{1}\frac{s^2\derivative s}{\sqrt{1-s^2}}
\]
Now setting $u=\arccos s$, so the interval becomes $(\arccos(\sqrt{R/(R+h)}),0)$
\end{frame}
\begin{frame}
$s=\cos u$, $\derivative s=-\sin u\derivative u$, $\int_{\sqrt{\frac{R}{R+h}}}^{1}\frac{s^2\derivative s}{\sqrt{1-s^2}}=\int_{0}^{\arccos(\sqrt{\frac{R}{(R+h)}})}\frac{\cos^2 u\sin u\derivative u}{\sin u}$
\begin{align*}
T&=\frac{2(R+h)^{3/2}}{\sqrt{2gR^2}}\int_{0}^{\arccos(\sqrt{\frac{R}{(R+h)}})}\cos^2 u\derivative u\\
&=\frac{(R+h)^{3/2}}{\sqrt{2gR^2}}\int_{0}^{\arccos(\sqrt{\frac{R}{(R+h)}})}(\cos (2u)+1)\derivative u\\
&=\frac{(R+h)^{3/2}}{2\sqrt{2gR^2}}\int_{0}^{2\arccos(\sqrt{\frac{R}{(R+h)}})}(\cos (w)+1)\derivative w\\
&=\frac{(R+h)^{3/2}}{2\sqrt{2gR^2}}\left[\sin\left(2\arccos(\sqrt{\frac{R}{(R+h)}})\right)+2\arccos(\sqrt{\frac{R}{(R+h)}})\right]\\
&=\frac{(R+h)^{3/2}}{\sqrt{2gR^2}}\left[\sqrt{\frac{h}{R+h}}\sqrt{\frac{R}{R+h}}+\arccos(\sqrt{\frac{R}{(R+h)}})\right]
\end{align*}
where we have exploited the identity $\sin 2\theta=2\sin\theta\cos\theta$.
\end{frame}
\begin{frame}
\frametitle{Critical Damp}
For a critically damped harmonic oscillator show that the oscillating mass can pass through the equilibrium position at most once, regardless of initial conditions.\\
The general solution to the critically damped harmonic oscillator is
\[x(t)=C_1e^{-\frac{b}{2m}t}+C_2te^{-\frac{b}{2m}t}=e^{-\frac{b}{2m}t}(C_1+C_2t)\]
$e^{-\frac{b}{2m}t}>0$, so there can only be one \emph{zero} at $C_1+C_2t=0$, so the critically damped harmonic oscillator can pass the equilibrium at most once.
\end{frame}
\end{document}
